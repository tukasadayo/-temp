\documentstyle[12pt,fleqn,epsbox,here,program]{myreport}
\makeatletter
\makeatother
%%% =====================================================
\thispagestyle{empty}
\setcounter{page}{0}
\begin{document}
%%%%%%%%%%%%%%%%%%%%%%%%%%%%%%%%%%%%%%%%%%%%%%%%%%%%%%%%%%%%%%%%%%%%%%%%%%
%%%%%%%%%%%%%%%%%%%%%%%%%%%%   タイトル   %%%%%%%%%%%%%%%%%%%%%%%%%%%%%%%%
%%%%%%%%%%%%%%%%%%%%%%%%%%%%%%%%%%%%%%%%%%%%%%%%%%%%%%%%%%%%%%%%%%%%%%%%%%
\begin{center}
平成25( 2013 )年度 卒業研究論文\\
%平成 12( 2000 )年度 修士研究論文\\
\vspace{1.5cm}
{\LARGE \mbox{ラットの行動時におけるシータ波の位相解析}} \\
%{\LARGE \mbox{汎用人型決戦兵器の作成とダミープラグ制御}}
\vspace{2.0cm}
提~~出~~者\\
\vspace{1.5cm}
{\Large 琉球大学工学部機械システム工学科}\\
%{\Large 琉球大学大学院理工学研究科}\\
%{\Large 博士前期課程機械システム工学専攻}\\
%{\Large 琉球大学大学院理工学研究科博士前期課程機械システム工学専攻}\\
\vspace{1.5cm}
{\Large 伊禮 司(125188B)}\\
\vspace{2.0cm}
指~~導~~教~~員\\
\vspace{0.5cm}
%金城 \quad 寛\\
%倉田 耕治\\
宮田 龍太\\
\vspace{1.5cm}
平成 28( 2016 )年 2月19日  提出\\
\end{center}
%%%%%%%%%%%%%%%%%%%%%%%%%%%%%%%%%%%%%%%%%%%%%%%%%%%%%%%%%%%%%%%%%%%%%%%%%%
\setlength{\baselineskip}{8mm}
\setlength{\headheight}{-1.5cm}
\def\thepage{\roman{page}}
\newpage
%%%%%%%%%%%%%%%%%%%%%%%%%%%%%%%%%%%%%%%%%%%%%%%%%%%%%%%%%%%%%%%%%%%%%%%%%%
%%%%%%%%%%%%%%%%%%%%%%%%%%%%   作者履歴   %%%%%%%%%%%%%%%%%%%%%%%%%%%%%%%%
%%%%%%%%%%%%%%%%%%%%%%%%%%%%%%%%%%%%%%%%%%%%%%%%%%%%%%%%%%%%%%%%%%%%%%%%%%
\begin{center}
{\Large\bf 平成25(2013)年度}\\
{\Large\bf 卒業論文提出者}\\
%{\Large\bf 修士論文提出者}\\

\vspace*{1cm}
\begin{minipage}[t]{15cm}
\begin{array}[t]{rcl}
 {\bf 氏名}& :&伊禮 司(イレイ ツカサ)\\
% {\bf 氏名}& :&パンスト 太郎(パンスト たろう)\\
 {\bf 学籍番号}& :& $125188B$\\
 {\bf 青年月日}& :& 平成 28(2016)年 2月19日\\
 {\bf 本籍}& :& $〒904-2161$~沖縄県沖縄市字古謝1143-3 \\
 {\bf 現住所}& :& $〒904-2161$~沖縄県沖縄市字古謝1143-3 \\
               && $TEL$:(098)-921-4666\\
 {\bf 保証人}& :& \\
\end{array}
\end{minipage}\\
\vspace*{1cm}
略歴\\
\begin{minipage}[t]{15cm}
\begin{array}[t]{cl}
2016年3月 & 琉球大学工学部機械システム工学科卒業見込\\
\end{array}
\end{minipage}\\
\end{center}
%\begin{figure}[H]
%\begin{center}
%\psbox[scale=0.5]{jpeg/penguin.eps}
%\end{center}
%\end{figure}
\TIMESTAMP{2016年 2月19日記入}
%%%%%%%%%%%%%%%%%%%%%%%%%%%%%%%%%%%%%%%%%%%%%%%%%%%%%%%%%%%%%%%%%%%%%%%%%%
%%%%%%%%%%%%%%%%%%%%%%%  目次と図目次生成コマンド  %%%%%%%%%%%%%%%%%%%%%%%
%%%%%%%%%%%%%%%%%%%%%%%%%%%%%%%%%%%%%%%%%%%%%%%%%%%%%%%%%%%%%%%%%%%%%%%%%%
\tableofcontents %目次
\listoffigures %図目次
%%%%%%%%%%%%%%%%%%%%%%%%%%%%%%%%%%%%%%%%%%%%%%%%%%%%%%%%%%%%%%%%%%%%%%%%%%
\pagebreak
\def\thepage{\arabic{page}}
\setcounter{page}{1}
\newpage
%%%%%%%%%%%%%%%%%%%%%%%%%%%%%%%%%%%%%%%%%%%%%%%%%%%%%%%%%%%%%%%%%%%%%%%%%%
%%%%%%%%%%%%%%%%%%%%%%%%%%%%  論文の本文  %%%%%%%%%%%%%%%%%%%%%%%%%%%%%%%%
%%%%%%%%%%%%%%%%%%%%%%%%%%%%%%%%%%%%%%%%%%%%%%%%%%%%%%%%%%%%%%%%%%%%%%%%%%
%%%%%%%%%%%%%%%%%%%%%%%%%%%%%%%%%%%%%%%%%%%%%%%%%%%%%%%%%%%%%%%%%%%%%%%%%%
%%------------------------------ 1章 ----------------------------------%%
%%%%%%%%%%%%%%%%%%%%%%%%%%%%%%%%%%%%%%%%%%%%%%%%%%%%%%%%%%%%%%%%%%%%%%%%%%
\chapter{はじめに}
%%%%%%%%%%%%%%%%%%%%%%%%%%%%%%%%%%%%%%%%%%%%%%%%%%%%%%%%%%%%%%%%%%%%%%%%%%
%%------------------------------ 2章 ----------------------------------%%
%%%%%%%%%%%%%%%%%%%%%%%%%%%%%%%%%%%%%%%%%%%%%%%%%%%%%%%%%%%%%%%%%%%%%%%%%%
\chapter{材料}
今回我々は、ラットを用いたレバー押し行動実験と、Linear maze行動実験のデータを用いて解析を行った。実験材料についてそれぞれ、次に説明する。
\section{ラットのレバー押し行動実験}

¥section{ラットのLinear maze行動実験}


%%%%%%%%%%%%%%%%%%%%%%%%%%%%%%%%%%%%%%%%%%%%%%%%%%%%%%%%%%%%%%%%%%%%%%%%%%
%%------------------------------ 3章 ----------------------------------%%
%%%%%%%%%%%%%%%%%%%%%%%%%%%%%%%%%%%%%%%%%%%%%%%%%%%%%%%%%%%%%%%%%%%%%%%%%%
\chapter{解析手法}
%%%%%%%%%%%%%%%%%%%%%%%%%%%%%%%%%%%%%%%%%%%%%%%%%%%%%%%%%%%%%%%%%%%%%%%%%%
%%------------------------------ 4章 ----------------------------------%%
%%%%%%%%%%%%%%%%%%%%%%%%%%%%%%%%%%%%%%%%%%%%%%%%%%%%%%%%%%%%%%%%%%%%%%%%%%
\chapter{結果}
%%%%%%%%%%%%%%%%%%%%%%%%%%%%%%%%%%%%%%%%%%%%%%%%%%%%%%%%%%%%%%%%%%%%%%%%%%
%%------------------------------ 5章 ----------------------------------%%
%%%%%%%%%%%%%%%%%%%%%%%%%%%%%%%%%%%%%%%%%%%%%%%%%%%%%%%%%%%%%%%%%%%%%%%%%%
\chapter{考察}
%%%%%%%%%%%%%%%%%%%%%%%%%%%%%%%%%%%%%%%%%%%%%%%%%%%%%%%%%%%%%%%%%%%%%%%%%%
%%------------------------------ 6章 ----------------------------------%%
%%%%%%%%%%%%%%%%%%%%%%%%%%%%%%%%%%%%%%%%%%%%%%%%%%%%%%%%%%%%%%%%%%%%%%%%%%
\chapter{結論}
\newpage
%%%%%%%%%%%%%%%%%%%%%%%%%%%%%%%%%%%%%%%%%%%%%%%%%%%%%%%%%%%%%%%%%%%%%%%%%%
%%%%%%%%%%%%%%%%%%%%%%%%%%%%%    謝辞    %%%%%%%%%%%%%%%%%%%%%%%%%%%%%%%%%
%%%%%%%%%%%%%%%%%%%%%%%%%%%%%%%%%%%%%%%%%%%%%%%%%%%%%%%%%%%%%%%%%%%%%%%%%%
\chapter*{謝 辞}
\addcontentsline{toc}{chapter}{謝 辞}
\vspace{12pt}
\hspace{19pc}\\平成 26 (2014) 年 2 月 19 日\\
\begin{center}
\vspace*{1cm}
~~~~~~~~~~~~~~~~~~~~~~~~~~~~~~~~~~~~~~~~~~~~~~~~~~~~~~~~\rule{60mm}{1pt}\\
%~~~~~~~~~\rule{8cm}{0.5pt}
\vspace*{2cm}
\end{center}
\newpage
%%%%%%%%%%%%%%%%%%%%%%%%%%%%%%%%%%%%%%%%%%%%%%%%%%%%%%%%%%%%%%%%%%%%%%%%%%
%%%%%%%%%%%%%%%%%%%%%%%%%%%%%  参考文献  %%%%%%%%%%%%%%%%%%%%%%%%%%%%%%%%%
%%%%%%%%%%%%%%%%%%%%%%%%%%%%%%%%%%%%%%%%%%%%%%%%%%%%%%%%%%%%%%%%%%%%%%%%%%
\addcontentsline{toc}{chapter}{参考文献}
\begin{thebibliography}{99}
\bibitem{book1}
O'Keefe J , Recce ML (1993) 
 Phase relationship between hippocampal place units and the EEG theta rhythm. 
 {\it Hippocampus} 3: 317 - 330. 
\bibitem{book2}
田中徳文,宮田龍太,青西亨,チャピ・ゲンツイ,臼井弘児,川原茂敬 (2015) 
Prediction of Lever Pressing with Hippocampal Theta Oscillation (II). 
第38回日本神経科学大会: 1P - 319.
\bibitem{book3}
Tanaka N et al. (2015) Neuroscience 2015 730.17. 
\bibitem{book4}
Mizuseki K et al. (2014)
Neurosharing: large-scale data sets (spike, LFP) recorded from the hippocampal-entorhinal system in behaving rats [version 1; referees: 4 approved]
 F1000Research 3:98.
\bibitem{book4} 
Mizuseki K et al. (2009) 
Theta oscillations provide temporal windows for local circuit computation in the entorhinal-hippocampal loop
Neuron 64(2):267-280. 
\bibitem{book5}
Pastalkova E et al. (2008) 
Internally generated cell assembly sequences in the rat hippocampus
Science 321(5894):1322-1327.
\bibitem{book6}
\end{thebibliography}
%%%%%%%%%%%%%%%%%%%%%%%%%%%%%%%%%%%%%%%%%%%%%%%%%%%%%%%%%%%%%%%%%%%%%%%%%%
%%%%%%%%%%%%%%%%%%%%%%%%%%%%%    付録    %%%%%%%%%%%%%%%%%%%%%%%%%%%%%%%%%
%%%%%%%%%%%%%%%%%%%%%%%%%%%%%%%%%%%%%%%%%%%%%%%%%%%%%%%%%%%%%%%%%%%%%%%%%%
\appendix
\end{document}
